\documentclass{article}

\usepackage[a4paper]{geometry}
\usepackage[francais]{babel}
\usepackage{amsmath}
\usepackage{xspace}

\newcommand{\mineq}{\min_{\text{eq}}\xspace}
\newcommand{\maxeq}{\max_{\text{eq}}\xspace}
\newcommand{\nproj}{N}
\newcommand{\nbeq}{k}
\newcommand{\eq}[1]{\text{eq}_{#1}}

\title{Algo}

\begin{document}

\maketitle

Au semestre prochain, dans le cadre d'une mati\`ere intitul\'ee
``Projet de programmation'', vous aurez le choix entre plusieurs
sujets de projet. Le but de ce document est de vous expliquer
l'algorithme utilis\'e pour la distribution de ces sujets. (C'est
\'egalement l'algorithme utilis\'e pour affelnet, le logiciel
d'affectation des lyc\'eens en seconde.)

Tout le cours de ``Pr\'epro 2: Conduite de projet'' sera construit
autour d'une application qui impl\'emente cet algorithme.

Pour plus de transparence, nous vous d\'ecrivons ici l'algorithme
utilis\'e et les enseignants de projet vous fourniront toutes les
donn\'ees pour que vous puissiez v\'erifier \emph{a posteriori\/} les
r\'esultats, mais l'algorithme en lui-m\^eme ne fait pas partie du
cours de conduite de projets.

\section{Situation initiale}
On d\'ecrit ici les donn\'ees fournies \`a l'algorithme, avec les
contraintes li\'ees \`a ces donn\'ees.
\paragraph{Les \'equipes}
Les projets sont \`a faire par \'equipe d'\'etudiants. La taille de
ces \'equipes est impos\'ee, entre \(\mineq\) et \(\maxeq\). On
dispose pour chaque membre de l'\'equipe de ses nom, pr\'enom et
adresse mail. Il y a \(\nbeq\) \'equipes.

\paragraph{Les v\oe ux}
Chaque \'equipe doit donner un ordre de pr\'ef\'erence sur l'ensemble
des sujets. Il y a \(\nproj\) sujets diff\'erents.

\paragraph{La graine de l'al\'ea}
Chaque \'equipe doit fournir un nombre entier strictement positif de
son choix, qui permettra de faire fonctionner l'algorithme
d'attribution des sujets.

\paragraph{Date limite}
Une date limite est fournie pour que les \'equipes s'inscrivent et
donnent leurs v\oe ux.

\section{L'algorithme}
L'algorithme se d\'eroule en plusieurs \'etapes (les \'etapes sont
expliqu\'ees en d\'etail plus bas):
\begin{enumerate}
\item on \'etablit un ordre initial;
\item on modifie l'ordre initial gr\^ace \`a l'ensemble des graines
  des \'equipes, pour obtenir une liste de d\'epart\footnote{Dans
    affelnet, la liste de d\'epart est li\'ee au nombre de points que
    vous avez en fonction de vos notes, votre statut, etc.};
\item on consid\`ere les \'equipes une par une, dans l'ordre de la
  liste de d\'epart \'etablie ci-dessus:
  \begin{itemize}
    \item pour l'\'equipe courante, on prend le sujet le mieux
      class\'e dans sa liste de v\oe ux pour lequel il reste de la
      place, et on le lui attribue.
  \end{itemize}
\end{enumerate}

\paragraph{Ordre initial}
L'ordre initial est obtenu en regardant l'adresse mail des membres de
l'\'equipe la plus petite dans l'ordre alphab\'etique (ordre
\texttt{ascii} pour comparer deux caract\`eres). Cet ordre est
total. On appelle \(\eq{1}\), \(\eq{2}\),\ldots ,\(\eq{\nbeq}\) les
\'equipes, dans l'ordre initial.

\paragraph{Liste de d\'epart}
Pour obtenir la liste de d\'epart, on applique une permutation \`a la
suite des \'equipes donn\'ee par l'ordre initial. Cette permutation
est obtenue sous forme de produit de \(\nbeq\) transpositions de
l'ensemble \(\{1,2,\ldots,\nbeq\}\), \`a partir des graines fournies
par les \'equipes (rappel: une transposition est une permutation qui
\'echange deux \'el\'ements et laisse les autres en place). Pour tout
\(i\) entre \(1\) et \(k\), on d\'efinit \(n_i\) comme \'etant la
somme des graines de toutes les \'equipes, sauf
l'\'equipe~\(\eq{i}\).

La transposition \(t_i\) inverse \(i\) et
\(i + \left(n_i\%(\nbeq-i+1)\right)\). Ce dernier est un entier entre
\(i\) et \(\nbeq\), donc \(t_i\) \'echange l'\'equipe en
position~\(i\)  avec une \'equipe en position sup\'erieure ou \'egale.

La permutation finale est obtenue par composition de ces transitions:
\(t_{\nbeq}\circ t_{2\nbeq-1}\circ \cdots\circ t_1\) (rappel: la
composition de fonctions s'effectue de droite \`a gauche).

On prend l'image de \((1,2,\ldots,\nbeq)\) par cette permutation pour
obtenir la liste de d\'epart. L'ordre de cette liste d\'epend donc des graines
donn\'ees par toutes les \'equipes (et des adresses mail de certains
\'etudiants), de fa\c con d\'eterministe, mais vous ne pouvez pas le
conna\^{\i}tre \emph{a priori} sans une communication totale entre
toutes les \'equipes.

\paragraph{\'Egalit\'e des chances?} On ne va pas se lancer dans une
analyse fine de cet algorithme, mais montrons que chaque \'equipe a
les m\^emes chances de se retrouver en premi\`ere position sur la
liste de d\'epart.

On suppose que les choix des graines par les \'equipes sont
al\'eatoires. On rappelle que \(t_i\)
\'echange l'\'equipe en position \(i\) avec une \'equipe en position
sup\'erieure ou \'egale \`a \(i\): une fois qu'on a appliqu\'e \(t_i\circ \cdots
\circ t_1\), le d\'ebut de la liste entre les positions~\(1\) et \(i\)
est fix\'e et ne peut plus \^etre modifi\'e par
\(t_\nbeq\circ\cdots\circ t_{i+1}\).

Il y a une seule fa\c con de se retrouver en premi\`ere position dans
la liste de d\'epart, c'est d'\^etre \'echang\'e avec \(1\) par
\(t_1\) (y compris pour~\(1\), dans ce cas \(t_1\) est
l'identit\'e). La probabilit\'e que cela arrive \`a un \'el\'ement est
donc de \(1/\nbeq\).

\medskip

Montrons que c'est la m\^eme probabilit\'e pour qu'une \'equipe se
retrouve en deuxi\`eme position dans la liste de d\'epart (et nous
nous arr\^eterons l\`a).

Pour que l'\'equipe~\(\eq{j}\) (en position \(j\) dans l'ordre initial)
se retrouve en deuxi\`eme position dans la liste de d\'epart, il faut
que sa position~\(i\) apr\`es l'application de \(t_1\) soit au
moins~\(2\): \(i\geq 2\) (probabilit\'e: \((\nbeq-1)/\nbeq\)), et que
\(t_2\) \'echange~\(2\) et~\(i\) (probabilit\'e: \(1/\nbeq\)). La
probabilit\'e que l'\'equipe~\(\eq{j}\) soit en deuxi\`eme position
dans la liste de d\'epart est donc \(1/\nbeq\).

\section{Exemple}

\end{document}
